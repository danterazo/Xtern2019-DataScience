\documentclass[]{article}
\usepackage{lmodern}
\usepackage{amssymb,amsmath}
\usepackage{ifxetex,ifluatex}
\usepackage{fixltx2e} % provides \textsubscript
\ifnum 0\ifxetex 1\fi\ifluatex 1\fi=0 % if pdftex
  \usepackage[T1]{fontenc}
  \usepackage[utf8]{inputenc}
\else % if luatex or xelatex
  \ifxetex
    \usepackage{mathspec}
  \else
    \usepackage{fontspec}
  \fi
  \defaultfontfeatures{Ligatures=TeX,Scale=MatchLowercase}
\fi
% use upquote if available, for straight quotes in verbatim environments
\IfFileExists{upquote.sty}{\usepackage{upquote}}{}
% use microtype if available
\IfFileExists{microtype.sty}{%
\usepackage{microtype}
\UseMicrotypeSet[protrusion]{basicmath} % disable protrusion for tt fonts
}{}
\usepackage[margin=1in]{geometry}
\usepackage{hyperref}
\hypersetup{unicode=true,
            pdftitle={Data Analysis for TechPointX Xbot Digital Assistant},
            pdfauthor={Dante Razo},
            pdfborder={0 0 0},
            breaklinks=true}
\urlstyle{same}  % don't use monospace font for urls
\usepackage{color}
\usepackage{fancyvrb}
\newcommand{\VerbBar}{|}
\newcommand{\VERB}{\Verb[commandchars=\\\{\}]}
\DefineVerbatimEnvironment{Highlighting}{Verbatim}{commandchars=\\\{\}}
% Add ',fontsize=\small' for more characters per line
\usepackage{framed}
\definecolor{shadecolor}{RGB}{248,248,248}
\newenvironment{Shaded}{\begin{snugshade}}{\end{snugshade}}
\newcommand{\AlertTok}[1]{\textcolor[rgb]{0.94,0.16,0.16}{#1}}
\newcommand{\AnnotationTok}[1]{\textcolor[rgb]{0.56,0.35,0.01}{\textbf{\textit{#1}}}}
\newcommand{\AttributeTok}[1]{\textcolor[rgb]{0.77,0.63,0.00}{#1}}
\newcommand{\BaseNTok}[1]{\textcolor[rgb]{0.00,0.00,0.81}{#1}}
\newcommand{\BuiltInTok}[1]{#1}
\newcommand{\CharTok}[1]{\textcolor[rgb]{0.31,0.60,0.02}{#1}}
\newcommand{\CommentTok}[1]{\textcolor[rgb]{0.56,0.35,0.01}{\textit{#1}}}
\newcommand{\CommentVarTok}[1]{\textcolor[rgb]{0.56,0.35,0.01}{\textbf{\textit{#1}}}}
\newcommand{\ConstantTok}[1]{\textcolor[rgb]{0.00,0.00,0.00}{#1}}
\newcommand{\ControlFlowTok}[1]{\textcolor[rgb]{0.13,0.29,0.53}{\textbf{#1}}}
\newcommand{\DataTypeTok}[1]{\textcolor[rgb]{0.13,0.29,0.53}{#1}}
\newcommand{\DecValTok}[1]{\textcolor[rgb]{0.00,0.00,0.81}{#1}}
\newcommand{\DocumentationTok}[1]{\textcolor[rgb]{0.56,0.35,0.01}{\textbf{\textit{#1}}}}
\newcommand{\ErrorTok}[1]{\textcolor[rgb]{0.64,0.00,0.00}{\textbf{#1}}}
\newcommand{\ExtensionTok}[1]{#1}
\newcommand{\FloatTok}[1]{\textcolor[rgb]{0.00,0.00,0.81}{#1}}
\newcommand{\FunctionTok}[1]{\textcolor[rgb]{0.00,0.00,0.00}{#1}}
\newcommand{\ImportTok}[1]{#1}
\newcommand{\InformationTok}[1]{\textcolor[rgb]{0.56,0.35,0.01}{\textbf{\textit{#1}}}}
\newcommand{\KeywordTok}[1]{\textcolor[rgb]{0.13,0.29,0.53}{\textbf{#1}}}
\newcommand{\NormalTok}[1]{#1}
\newcommand{\OperatorTok}[1]{\textcolor[rgb]{0.81,0.36,0.00}{\textbf{#1}}}
\newcommand{\OtherTok}[1]{\textcolor[rgb]{0.56,0.35,0.01}{#1}}
\newcommand{\PreprocessorTok}[1]{\textcolor[rgb]{0.56,0.35,0.01}{\textit{#1}}}
\newcommand{\RegionMarkerTok}[1]{#1}
\newcommand{\SpecialCharTok}[1]{\textcolor[rgb]{0.00,0.00,0.00}{#1}}
\newcommand{\SpecialStringTok}[1]{\textcolor[rgb]{0.31,0.60,0.02}{#1}}
\newcommand{\StringTok}[1]{\textcolor[rgb]{0.31,0.60,0.02}{#1}}
\newcommand{\VariableTok}[1]{\textcolor[rgb]{0.00,0.00,0.00}{#1}}
\newcommand{\VerbatimStringTok}[1]{\textcolor[rgb]{0.31,0.60,0.02}{#1}}
\newcommand{\WarningTok}[1]{\textcolor[rgb]{0.56,0.35,0.01}{\textbf{\textit{#1}}}}
\usepackage{graphicx,grffile}
\makeatletter
\def\maxwidth{\ifdim\Gin@nat@width>\linewidth\linewidth\else\Gin@nat@width\fi}
\def\maxheight{\ifdim\Gin@nat@height>\textheight\textheight\else\Gin@nat@height\fi}
\makeatother
% Scale images if necessary, so that they will not overflow the page
% margins by default, and it is still possible to overwrite the defaults
% using explicit options in \includegraphics[width, height, ...]{}
\setkeys{Gin}{width=\maxwidth,height=\maxheight,keepaspectratio}
\IfFileExists{parskip.sty}{%
\usepackage{parskip}
}{% else
\setlength{\parindent}{0pt}
\setlength{\parskip}{6pt plus 2pt minus 1pt}
}
\setlength{\emergencystretch}{3em}  % prevent overfull lines
\providecommand{\tightlist}{%
  \setlength{\itemsep}{0pt}\setlength{\parskip}{0pt}}
\setcounter{secnumdepth}{0}
% Redefines (sub)paragraphs to behave more like sections
\ifx\paragraph\undefined\else
\let\oldparagraph\paragraph
\renewcommand{\paragraph}[1]{\oldparagraph{#1}\mbox{}}
\fi
\ifx\subparagraph\undefined\else
\let\oldsubparagraph\subparagraph
\renewcommand{\subparagraph}[1]{\oldsubparagraph{#1}\mbox{}}
\fi

%%% Use protect on footnotes to avoid problems with footnotes in titles
\let\rmarkdownfootnote\footnote%
\def\footnote{\protect\rmarkdownfootnote}

%%% Change title format to be more compact
\usepackage{titling}

% Create subtitle command for use in maketitle
\newcommand{\subtitle}[1]{
  \posttitle{
    \begin{center}\large#1\end{center}
    }
}

\setlength{\droptitle}{-2em}

  \title{Data Analysis for TechPointX Xbot Digital Assistant}
    \pretitle{\vspace{\droptitle}\centering\huge}
  \posttitle{\par}
    \author{Dante Razo}
    \preauthor{\centering\large\emph}
  \postauthor{\par}
      \predate{\centering\large\emph}
  \postdate{\par}
    \date{10/20/2018}

\usepackage{amsmath}

\begin{document}
\maketitle

\hypertarget{abstract}{%
\section{Abstract}\label{abstract}}

I've been tasked with analyzing app store data to assess its current
state and predict the success of TechPointX's upcoming \(\textbf{Xbot}\)
digital assistant. Due to the popularity of the company's OSXtern
operating system, expectations are high for the app. I observed trends
in the app store to help the team make the launch as impactful as
possible. I used \(\textit{RMarkdown}\), \(\textit{RStudio}\), and the
\(\textit{DataExplorer}\) library for R to create this report.

\hypertarget{importing-data-packages}{%
\subsection{Importing Data \& Packages}\label{importing-data-packages}}

First, I imported the data. The \(\textit{.csv}\) has headers and
entries are separated by commas, so I used \(\texttt{read.csv()}\)'s
default settings.

\begin{Shaded}
\begin{Highlighting}[]
\KeywordTok{require}\NormalTok{(DataExplorer)   }\CommentTok{# package that provides additional visualization tools for data analysis}
\end{Highlighting}
\end{Shaded}

\begin{verbatim}
## Loading required package: DataExplorer
\end{verbatim}

\begin{Shaded}
\begin{Highlighting}[]
\NormalTok{appStore <-}\StringTok{ }\KeywordTok{read.csv}\NormalTok{(}\DataTypeTok{file=}\StringTok{"AppStoreAssessmentDataScience.csv"}\NormalTok{)}
\NormalTok{appStore.og <-}\StringTok{ }\NormalTok{appStore }\CommentTok{# store copy of the original before preprocessing}
\end{Highlighting}
\end{Shaded}

\hypertarget{data-preprocessing}{%
\subsection{Data Preprocessing}\label{data-preprocessing}}

Before any preprocessing is done, we can observe that
\(\textit{appStore}\) contains 7197 objects of 9-dimensions. There are
no missing values, so imputation is not necessary.

\begin{Shaded}
\begin{Highlighting}[]
\KeywordTok{dim}\NormalTok{(appStore)}
\end{Highlighting}
\end{Shaded}

\begin{verbatim}
## [1] 7197    9
\end{verbatim}

\begin{Shaded}
\begin{Highlighting}[]
\KeywordTok{sum}\NormalTok{(}\KeywordTok{is.na}\NormalTok{(appStore)) }\CommentTok{# no missing values in dataset}
\end{Highlighting}
\end{Shaded}

\begin{verbatim}
## [1] 0
\end{verbatim}

The first column of \(\textit{appStore}\) is in numerical order, but
only the first 18 entries match the column number. It's unknown why
numbers are skipped over. This vector has a 99\% correlation with column
numbers, so I removed it from the dataset. It is stored under a new name
in case it can be used later.

\begin{Shaded}
\begin{Highlighting}[]
\KeywordTok{sum}\NormalTok{(appStore[}\DecValTok{1}\NormalTok{] }\OperatorTok{==}\StringTok{ }\KeywordTok{seq}\NormalTok{(}\DecValTok{1}\NormalTok{,}\KeywordTok{nrow}\NormalTok{(appStore))) }\CommentTok{# checks if entry equals row number; 18 matches}
\end{Highlighting}
\end{Shaded}

\begin{verbatim}
## [1] 18
\end{verbatim}

\begin{Shaded}
\begin{Highlighting}[]
\KeywordTok{cor}\NormalTok{(appStore}\OperatorTok{$}\NormalTok{X, }\KeywordTok{seq}\NormalTok{(}\DecValTok{1}\NormalTok{,}\KeywordTok{nrow}\NormalTok{(appStore)))    }\CommentTok{# computes correlation between two vectors}
\end{Highlighting}
\end{Shaded}

\begin{verbatim}
## [1] 0.9936812
\end{verbatim}

\begin{Shaded}
\begin{Highlighting}[]
\NormalTok{appStore.V1 <-}\StringTok{ }\NormalTok{appStore[}\DecValTok{1}\NormalTok{]                }\CommentTok{# save first column as new variable}
\NormalTok{appStore <-}\StringTok{ }\NormalTok{appStore[,}\DecValTok{2}\OperatorTok{:}\DecValTok{9}\NormalTok{]                }\CommentTok{# remove first column from dataset}
\end{Highlighting}
\end{Shaded}

The \(\textit{app\_content\_rating}\) column contains integers with a
``+'' character appended to the end. I removed the plusses and converted
the resulting strings to integers. This will allow me to take averages
and analyze this vector in the future.

\begin{Shaded}
\begin{Highlighting}[]
\NormalTok{appStore}\OperatorTok{$}\NormalTok{app_content_rating <-}\StringTok{ }\KeywordTok{as.numeric}\NormalTok{(}\KeywordTok{gsub}\NormalTok{(}\StringTok{'}\CharTok{\textbackslash{}\textbackslash{}}\StringTok{+'}\NormalTok{,}\StringTok{''}\NormalTok{, appStore}\OperatorTok{$}\NormalTok{app_content_rating))}
\end{Highlighting}
\end{Shaded}

It was at this point that I remembered to check for other types of
missing values (such as zeroes where they don't make sense). Using the
\(\texttt{summary()}\) function revealed that the last column of the
dataset (\(\textit{app\_total\_supported\_langs}\)) contained 0's. It
doesn't make sense for an app to have 0 supported languages, so these
are effectively missing values. Due to the small number of affected
entries, I elected to simply remove them.

\begin{Shaded}
\begin{Highlighting}[]
\KeywordTok{summary}\NormalTok{(appStore}\OperatorTok{$}\NormalTok{app_total_supported_langs)}
\end{Highlighting}
\end{Shaded}

\begin{verbatim}
##    Min. 1st Qu.  Median    Mean 3rd Qu.    Max. 
##   0.000   1.000   1.000   5.435   8.000  75.000
\end{verbatim}

\begin{Shaded}
\begin{Highlighting}[]
\KeywordTok{sum}\NormalTok{(appStore}\OperatorTok{$}\NormalTok{app_total_supported_langs }\OperatorTok{==}\StringTok{ }\DecValTok{0}\NormalTok{)}
\end{Highlighting}
\end{Shaded}

\begin{verbatim}
## [1] 41
\end{verbatim}

\begin{Shaded}
\begin{Highlighting}[]
\NormalTok{appStore}\OperatorTok{$}\NormalTok{app_total_supported_langs[appStore}\OperatorTok{$}\NormalTok{app_total_supported_langs }\OperatorTok{==}\StringTok{ }\DecValTok{0}\NormalTok{] <-}\StringTok{ }\OtherTok{NA} \CommentTok{# replace 0's with NA}
\NormalTok{appStore <-}\StringTok{ }\KeywordTok{na.omit}\NormalTok{(appStore)}
\end{Highlighting}
\end{Shaded}

\hypertarget{data-analysis}{%
\subsection{Data Analysis}\label{data-analysis}}

Now that the data has been processed, we can begin to make sense of it.
I begin by making density plots of every column in the dataset. The
\(\textit{DataExplorer}\) library makes it easy to view all plots at
once.

\begin{Shaded}
\begin{Highlighting}[]
\KeywordTok{plot_density}\NormalTok{(appStore) }\CommentTok{# function from DataExplorer library}
\end{Highlighting}
\end{Shaded}

\includegraphics{xtern_ds_report_files/figure-latex/analysis1-1.pdf}

I immediately noticed that \(\texttt{appStore\$id}\) is a a left-skewed
bimodal distribution. The ID is simply a number and won't be useful in
identifying trends in the App Store, so I moved on to other columns. The
majority of apps on the market are less than 1000MB (\$10\^{}9 =
1,000,000,000 \$bytes). Depending on how OSXtern and other supported
platforms defines a gigabyte, you could say that most apps are less than
1GB as well.

\begin{Shaded}
\begin{Highlighting}[]
\KeywordTok{plot_correlation}\NormalTok{(appStore[,}\DecValTok{2}\OperatorTok{:}\KeywordTok{ncol}\NormalTok{(appStore)], }\DataTypeTok{maxcat=}\DecValTok{12}\NormalTok{) }\CommentTok{# function from DataExplorer library}
\end{Highlighting}
\end{Shaded}

\begin{verbatim}
## Warning in dummify(data, maxcat = maxcat): Ignored all discrete features
## since `maxcat` set to 12 categories!
\end{verbatim}

\includegraphics{xtern_ds_report_files/figure-latex/analysis2-1.pdf}

\(\textit{DataExplorer}\)'s \(\texttt{plot\_correlation()}\) function
produced a nice correlation graph of every feature. I didn't learn
anything new from it, but it confirmed that no high-correlation vector
pairs remained. The only feature not pictured is \(\texttt{app\_genre}\)
because each entry is a string. It contains nominal data; usually, I'd
assign numbers to each value before working with them, but
\(\textit{DataExplorer}\) has a visualization solution that negates the
need to first quantify the genres.

\begin{Shaded}
\begin{Highlighting}[]
\KeywordTok{plot_bar}\NormalTok{(appStore}\OperatorTok{$}\NormalTok{app_genre, }\DataTypeTok{title=}\StringTok{"Frequency of Genres in the App Store"}\NormalTok{)}
\end{Highlighting}
\end{Shaded}

\includegraphics{xtern_ds_report_files/figure-latex/analysis3-1.pdf}

The `Games' category stands out as an outlier. The number of `Games'
apps (3832) is over 7 times gerater than the number of `Entertainment'
apps (534). I created a separate dataset that contains everything but
apps labeled `Games' in case the outlier affects future observations.

\begin{Shaded}
\begin{Highlighting}[]
\NormalTok{numGames <-}\StringTok{ }\KeywordTok{sum}\NormalTok{(}\KeywordTok{grepl}\NormalTok{(}\StringTok{"Games"}\NormalTok{, appStore}\OperatorTok{$}\NormalTok{app_genre))                 }\CommentTok{# most common genre}
\NormalTok{numEntertainment <-}\StringTok{ }\KeywordTok{sum}\NormalTok{(}\KeywordTok{grepl}\NormalTok{(}\StringTok{"Entertainment"}\NormalTok{, appStore}\OperatorTok{$}\NormalTok{app_genre)) }\CommentTok{# second most common genre}
\NormalTok{numGames }\OperatorTok{/}\StringTok{ }\NormalTok{numEntertainment                                         }\CommentTok{# ratio (7x increase)}
\end{Highlighting}
\end{Shaded}

\begin{verbatim}
## [1] 7.17603
\end{verbatim}

\begin{Shaded}
\begin{Highlighting}[]
\NormalTok{appStore.noGames <-}\StringTok{ }\NormalTok{appStore                                        }\CommentTok{# create copy of dataset}
\NormalTok{appStore.noGames}\OperatorTok{$}\NormalTok{app_genre[}\KeywordTok{grepl}\NormalTok{(}\StringTok{"Games"}\NormalTok{, appStore.noGames}\OperatorTok{$}\NormalTok{app_genre)] <-}\StringTok{ }\OtherTok{NA} \CommentTok{# replace games with NA}
\NormalTok{appStore.noGames <-}\StringTok{ }\KeywordTok{na.omit}\NormalTok{(appStore)                               }\CommentTok{# remove games (now NA)}
\end{Highlighting}
\end{Shaded}

\(\textbf{Xbot}\) is an assistant, so it'd best fit in the `Utilities'
category. I compared this category to its nearest competitors below:

\begin{Shaded}
\begin{Highlighting}[]
\NormalTok{numPhoto  <-}\StringTok{ }\KeywordTok{sum}\NormalTok{(}\KeywordTok{grepl}\NormalTok{(}\StringTok{"Photo & Video"}\NormalTok{, appStore}\OperatorTok{$}\NormalTok{app_genre))}
\NormalTok{numUtil   <-}\StringTok{ }\KeywordTok{sum}\NormalTok{(}\KeywordTok{grepl}\NormalTok{(}\StringTok{"Utilities"}\NormalTok{, appStore}\OperatorTok{$}\NormalTok{app_genre))}
\NormalTok{numHealth <-}\StringTok{ }\KeywordTok{sum}\NormalTok{(}\KeywordTok{grepl}\NormalTok{(}\StringTok{"Health & Fitness"}\NormalTok{, appStore}\OperatorTok{$}\NormalTok{app_genre))}

\NormalTok{numUtil                          }\CommentTok{# number of utility apps}
\end{Highlighting}
\end{Shaded}

\begin{verbatim}
## [1] 248
\end{verbatim}

\begin{Shaded}
\begin{Highlighting}[]
\NormalTok{numPhoto }\OperatorTok{-}\StringTok{ }\NormalTok{numUtil               }\CommentTok{# distance to upper neighbor}
\end{Highlighting}
\end{Shaded}

\begin{verbatim}
## [1] 100
\end{verbatim}

\begin{Shaded}
\begin{Highlighting}[]
\NormalTok{numUtil }\OperatorTok{-}\StringTok{ }\NormalTok{numHealth              }\CommentTok{# distance to lower neighbor}
\end{Highlighting}
\end{Shaded}

\begin{verbatim}
## [1] 68
\end{verbatim}

Unlike `Games', `Utilities' is reasonably close to its neighbors. It's a
popular yet less-saturated genre with only 248 apps. \(\textbf{Xbot}\)
has a higher chance of success than any new game that comes to the app
store because it has less competition. If advertised properly, it could
very easily top the charts.

Next, I focused on the price of the apps in the dataset
(\(\texttt{appStore\$app\_price}\)). Unsurprisingly, the majority (56\%)
were free. I must admit that I'm unsure what to make of this; the naive
answer would be to make \(\textbf{Xbot}\) a free app too to achieve the
same accessibility and popularity as the others. People are more likely
to download and try a free app than pay for an app they may not enjoy
using.

\begin{Shaded}
\begin{Highlighting}[]
\KeywordTok{sum}\NormalTok{(appStore}\OperatorTok{$}\NormalTok{app_price }\OperatorTok{==}\StringTok{ }\FloatTok{0.00}\NormalTok{) }\OperatorTok{/}\StringTok{ }\KeywordTok{nrow}\NormalTok{(appStore)            }\CommentTok{# 56% of apps are free}
\end{Highlighting}
\end{Shaded}

\begin{verbatim}
## [1] 0.5627446
\end{verbatim}

\begin{Shaded}
\begin{Highlighting}[]
\NormalTok{appStore.prices <-}\StringTok{ }\KeywordTok{as.data.frame}\NormalTok{(}\KeywordTok{table}\NormalTok{(appStore}\OperatorTok{$}\NormalTok{app_price)) }\CommentTok{# store prices in new dataframe}
\KeywordTok{names}\NormalTok{(appStore.prices)[}\DecValTok{1}\NormalTok{] <-}\StringTok{ "Price"}                        \CommentTok{# rename first dataframe vector}
\KeywordTok{plot}\NormalTok{(appStore.prices)                                       }\CommentTok{# plot distribution of prices}
\end{Highlighting}
\end{Shaded}

\includegraphics{xtern_ds_report_files/figure-latex/analysis6-1.pdf}

\hypertarget{conclusion}{%
\section{Conclusion}\label{conclusion}}

To make \(\textbf{Xbot}\) a success, TechPointX needs to list the app as
a ``Utility'' and consider making the app free to incite downloads. As a
courtesy to consumers, an app size less than 1GB is ideal.


\end{document}
